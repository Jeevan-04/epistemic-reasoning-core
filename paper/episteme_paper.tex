\documentclass[10pt,twocolumn,letterpaper]{article}

\usepackage[utf8]{inputenc}
\usepackage{graphicx}
\usepackage{amsmath}
\usepackage{amssymb}
\usepackage{hyperref}
\usepackage{geometry}
\usepackage{booktabs}
\usepackage{float}

% Margins
\geometry{top=0.75in, bottom=0.75in, left=0.65in, right=0.65in}

\title{\textbf{Episteme: A Persistent Epistemic Reasoning Core for \\ Robust Symbolic Logic and Quantitative Belief Management}}
\author{Jeevan \\ \textit{MARC Project}}
\date{January 2026}

\begin{document}

\maketitle

\begin{abstract}
As Large Language Models (LLMs) continue to dominate artificial intelligence, the "Hallucination Problem" remains a critical barrier to reliability. This paper introduces \textbf{Episteme}, a research-grade epistemic reasoning system designed to decouple logical validity from probabilistic likelihood. Episteme implements the \textit{Modular Architecture for Reasoning and Cognition} (MARC), strictly separating the acquisition of knowledge (Manas), its quantitative persistence (Chitta), and its logical processing (Buddhi). By enforcing a "Logic First, Numbers Second" philosophy, Episteme achieves 100\% accuracy on compositional logic benchmarks and demonstrates robust handling of defeasible inheritance (the "Penguin Problem") and horizontal ambiguity (the "Nixon Diamond"). We present the architectural specifications, the quantitative belief formulas, and evaluation results across 1,050 adversarial test cases.
\end{abstract}

\vspace{0.5em}
\noindent\textbf{Keywords:} Epistemic Logic, Non-Monotonic Reasoning, MARC, Neuro-Symbolic AI, Knowledge Graphs.

\section{Introduction}

The integration of symbolic logic with neural networks typically takes the form of "neuro-symbolic" hybrids, where logical constraints are often softened into loss functions or vector embeddings. While this improves flexibility, it compromises reliability; a system that views $A \implies B$ as "mostly true" cannot perform rigorous multi-step deduction without drift.

Episteme takes a divergent approach: \textit{Architectural Separation}. It posits that the mechanism for \textit{understanding} language (Acquisition) must be distinct from the mechanism for \textit{verifying} truth (Reasoning). The Core axiom of the system is: \textbf{"Logic determines Validity; Numbers determine Availability."}

This paper details the implementation of this philosophy through the Episteme system, focusing on its three primary layers: the Acquisition layer (Manas), the Storage layer (Chitta), and the Logic layer (Buddhi).

\section{System Architecture}

Episteme is built upon the MARC design pattern, ensuring no single component manages both perception and reasoning.

\subsection{Manas: The Acquisition Layer}
Manas ("The Mind") serves as the stateless sensorium. Its sole responsibility is to parse natural language into structured, normalized \textit{Belief Proposals}. It does not perform belief updates or reasoning.

The parsing pipeline consists of three stages:
\begin{enumerate}
    \item \textbf{Intent Detection:} Classifying input as Assertion ("Socrates is a man") or Query ("Is Socrates a man?").
    \item \textbf{Entity Normalization:} Mapping linguistic variations (e.g., "The Human", "humans") to canonical forms (`human`).
    \item \textbf{Strict Sanitation:} A "logical immune system" that rejects non-semantic tokens. Numeric entities (e.g., "1") and leaked verbs ("active") are strictly pruned to prevent graph pollution.
\end{enumerate}

\subsection{Chitta: The Quantitative Memory}
Chitta ("Memory") is the persistent hypergraph storage. Unlike standard graph databases, Chitta manages the \textit{lifecycle} of a belief using quantitative parameters. It does not decide truth; it decides \textit{visibility}.

A belief $B$ is defined as a tuple:
$$ B = \langle E_s, P, E_o, \tau, C, N \rangle $$
Where $\tau$ is the Epistemic Type (AXIOM, DEFAULT, etc.), $C$ is Confidence, and $N$ is Evidence Count.

\subsection{Buddhi: The Logic Engine}
Buddhi ("Intellect") is the pure inference engine. It operates on the state provided by Chitta but applies a rigorous \textit{Lattice of Truth}. It supports non-monotonic reasoning, allowing specific exceptions to override general rules (Defeasible Logic).

\section{Quantitative Dynamics}

While Logic is binary or ternary, the \textit{availability} of knowledge is continuous. Episteme models this via Reinforcement and Decay.

\subsection{Reinforcement}
When an existing belief is restated by a source, its confidence $C$ is boosted asymptotically towards 1.0. The update rule is:
\begin{equation}
    C_{t+1} = C_t + (1 - C_t) \cdot \alpha
\end{equation}
Where $\alpha$ is the learning rate (typically 0.05). This ensures that repeated observation increases certainty without exceeding unity.

\subsection{Temporal Decay}
To simulate the volatility of unverified information, beliefs decay over time steps $t$:
\begin{equation}
    C_t = C_0 \cdot D^t
\end{equation}
Where $D$ is the decay rate (e.g., 0.995).

\subsection{Logic Gating}
The interface between Chitta and Buddhi is governed by a gating threshold $\theta$:
\begin{equation}
    Status(B) = 
    \begin{cases} 
    \text{ACTIVE} & \text{if } C > \theta \\
    \text{INACTIVE} & \text{if } C \leq \theta 
    \end{cases}
\end{equation}
Inactive beliefs remain in storage but are invisible to the inference engine, effectively simulating "forgetting" without data loss.

\section{Logic and Inference}

Buddhi employs a hierarchical verification system to resolve contradictions.

\subsection{The Lattice of Truth}
Beliefs are ranked by epistemic weight:
\begin{enumerate}
    \item \textbf{AXIOM:} Immutable constraints.
    \item \textbf{OBSERVATION:} Empirical facts.
    \item \textbf{EXCEPTION:} Specific overrides.
    \item \textbf{DEFAULT:} General rules.
\end{enumerate}
$$ \text{AXIOM} \succ \text{OBSERVATION} \succ \text{EXCEPTION} \succ \text{DEFAULT} $$

\subsection{Conflict Resolution Matrix}
When two valid paths lead to contradictory conclusions ($A$ and $\neg A$), the verdict is determined by structural properties:

\begin{itemize}
    \item \textbf{Vertical Conflict:} If Path $A$ relies on a subclass (Specificity) and Path $B$ on a superclass, specific wins. (e.g., Penguin > Bird).
    \item \textbf{Horizontal Conflict:} If Path $A$ and Path $B$ have equal Rank and Distance, the result is the \textbf{Nixon Diamond}, yielding a verdict of \texttt{CONFLICT}.
\end{itemize}

\section{Evaluation}

The system was evaluated against a "Brutal Benchmark" suite of 1,050 adversarial test cases.

\begin{table}[H]
\centering
\caption{Benchmark Accuracy by Category}
\label{tab:benchmarks}
\begin{tabular}{@{}lcc@{}}
\toprule
\textbf{Category} & \textbf{Cases} & \textbf{Accuracy} \\ \midrule
Compositional Logic & 70 & 100.0\% \\
Ungrounded Refusal & 150 & 100.0\% \\
Entity Ambiguity & 50 & 100.0\% \\
Cross-Frame Isolation & 150 & 98.0\% \\
Explicit Contradiction & 350 & 74.6\% \\
Inheritance Exception & 150 & 60.7\% \\ \midrule
\textbf{Overall} & \textbf{1050} & \textbf{84.7\%} \\ \bottomrule
\end{tabular}
\end{table}

\subsection{Analysis}
The system achieves perfect scores in compositional logic, validating the Transitive Inference engine. The 100\% score in Ungrounded Refusal highlights the system's "Epistemic Humility"—it correctly responds \texttt{UNKNOWN} rather than hallucinating. The lower score in Inheritance Exception (60.7\%) reflects the immense complexity of defeasible reasoning in natural language, though it represents a significant improvement over baseline vector-retrieval methods which typically score near random on specific overrides.

\section{Conclusion}

Episteme demonstrates that rigorous symbolic logic can be effectively bolstered by quantitative lifecycle management. By distinguishing between the \textit{truth} of a statement (Logic) and the \textit{strength} of a memory (Chitta), we achieve a system that is robust against hallucination, capable of specific overrides, and logically consistent. Future work will focus on expanding the relation algebra to include temporal reasoning and causal inference.

\end{document}
